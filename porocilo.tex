\documentclass[a4paper, 10pt]{article}


\usepackage[slovene]{babel}
\usepackage[utf8]{inputenc}
\usepackage[T1]{fontenc}
\usepackage{amsmath}
\usepackage{graphicx}
\usepackage{listings}



\begin{document}
\newtheorem{example}{Zgled}
\newtheorem{theorem}{Izrek}
\newtheorem{proof}{Dokaz}
\newtheorem{definition}{Definicija}
\newtheorem{lemma}{Lema}

\thispagestyle{empty}
{\large
\noindent Univerza v Ljubljani\\[1mm]
Fakulteta za matematiko in fiziko\\[5mm]}
\vfill

\begin{center}{\Large
{\bf Odpornost snarkov}\\[2mm]
Vanja Kalaković in Eva Strašek\\[10mm]
Mentorja: Janoš Vidali, Riste Škrekovski\\[2mm]
Predmet: Finančna matematika \\[2mm]}
\end{center}
\vfill

{\large
Ljubljana, 2023}
\pagebreak



\section{Uvod}

Najprej bomo opredelili, kaj snarki sploh so.

\begin{definition}
    Če so vsa vozlišča grafa G enake stopnje k, pravimo, da je graf
k-regularen; 3-regularnim grafom pravimo tudi kubični grafi.
\end{definition}
\begin{definition}
    Kromatični indeks grafa $G$ je število $\chi'(G)$, ki predstavlja najmanjše 
    število barv, ki jih potrebujemo za barvanje povezav grafa G tako,
    da so sosednje povezave pobarvane različno.
\end{definition}
\begin{definition}
    Snark je ciklično 4-povezan kubičen graf z notranjim obsegom vsaj 5 in 
    kromatičnim indeksom 4. Snark reda $n$ in velikosti $m$ označimo z $S(n,m)$.
\end{definition}

 
\subsection*{Zgodovina snarkov}
Snark je svoje ime dobil leta 1976, ko ga je Markin Gardner poimenoval po skrivnostnem 
predmetu iz pesmi The Hunting of the Snark Lewisa Carrolla. \\
Prvi znani snark, Petersenov graf, je odkril Julius Petersen in zanj velja, da ima 10 vozlišč
in 15 povezav ter je najmanjši snark. V naslednjih letih so matematiki odrili nekaj manjših
snarkov, dokler ni do leta 1975 ameriški matematik Isaacs posplošil metodo, hrvaškega
matematika Blanuša, za izgradnjo dveh neskončnih družin snarkov, in sicer za družino
snarkov v obliki cvetja in za Blanuša-Descartes-Szekersov snark. Isaacs je odkril tudi snark 
poimenovan dvojna zvezda.  

\subsection*{Odpornost povezav in vozlišč}

\begin{theorem}{Vizingov izrek}
    Za enostaven graf G z maksimalno stopnjo $\Delta$(G) velja
    $\Delta$(G) $\leq \chi$'(G) $\leq \Delta$(G) + 1. 
\end{theorem}

\begin{definition}
    Predpostavimo, da imamo graf $G = (V,E)$. Odpornost povezav $er(G)$ je je najmanjše 
    število $k$ povezav, ki jih je treba odstraniti, da postane graf $G$ $k$-robno obarljiv. 
\end{definition}

Koda, s katero sma računali odpornost povezav snarkov:
\begin{lstlisting}
    def odpornost_povezav(graf):
        seznam_odpornosti = list(zero_vector(60))
        for i in range(60):
            graf1 = copy(graf)
            odpornost = 0
            while graf1.chromatic_index() != 3:
                u,v = graf1.random_edge(labels=False)
                graf1.delete_edge(u,v)
                odpornost += 1
            seznam_odpornosti[i] = odpornost
        return min(seznam_odpornosti)
\end{lstlisting}
\pagebreak

\begin{definition}
    Predpostavimo, da imamo graf $G = (V,E)$. Odpornost vozlišč $vr(G)$ je najmanjše število $k$
    vozlišč $E$, ki jih je treba odstraniti, da postane graf $G$ $k$-robno obarljiv.
\end{definition}

Koda, s katero sma izračunali odpornost vozlišč snarkov: 
\begin{lstlisting}
    def odpornost_vozlisc(graf):
        seznam_odpornosti = list(zero_vector(30))
        for i in range(30):
            graf1 = copy(graf)
            odpornost = 0
            while graf1.chromatic_index() != 3:
                u = graf1.random_vertex()
                graf1.delete_vertex(u)
                odpornost += 1
                seznam_odpornosti[i] = odpornost
        return min(seznam_odpornosti)
\end{lstlisting}

\subsection*{Predstavitev rezultatov}

V spodnji tabeli vidimo odpornosti povezav in odpornosti vozlišč kot sva
jih izračunali s pomočjo najine kode.

\begin{center}
    \begin{tabular}{|c|c|c|c|c|}
        \hline
        snark & št. vozl. & št. povez. & odpor. povez. & odpor. vozl. \\
        \hline
        Petersenov graf & 10 & 15 & 2 & 2 \\ 
        \hline
        Graf 2760 & 18 & 27 & 2 & 2 \\
        \hline
        Flower snark J5 & 20 & 30 & 2 & 2 \\
        \hline
        Loupekines snark 1 & 22 & 33 & 2 & 2 \\
        \hline
        Loupekines snark 2 & 22 & 33 & 2 & 2 \\ 
        \hline
        Golberg snark 3 & 24 & 36 & 2 & 2 \\
        \hline
        Celmins Swart snark 1 & 26 & 39 & 2 & 2\\
        \hline 
        Celmins Swart snark 2 & 26 & 39 & 2 & 2\\
        \hline
        Flower snark J5 & 28 & 42 & 2 & 2 \\
        \hline
        Double star snark & 30 & 35 & 2 & 2 \\
        \hline
        Graf 3337 & 34 & 51 & 2 & 2 \\
        \hline
        Graf 3363 & 36 & 54 & 2 & 2 \\
        \hline
        Golberg snark 5 & 40 & 60 & 2 & 2 \\
        \hline
        Graf 25159 & 44 & 66 & 3 & 3 \\
        \hline
    \end{tabular}
    \end{center}

Ugotovili sva, da velja da za vsak poljuben snark velja, da je njegova $vr(G) \ge 2$ in
njegova $er(G) \ge 2$. Poleg tega velja $er(G) \ge vr(G) $ Ugotovili sva, da so najmanjši
snarki, katerih $vr(G) \ge 3$ snarki z 44 vozlišči in 66 povezavami.




\pagebreak
\subsection*{Potek dela:}
Najprej sva napisali program, ki je izračunal odpornost vozlišč $vr(G)$ in program, ki je 
izračunal odpornost povezav $er(G)$ za snarke, ki sva jih prenesli iz House of Graphs.
Programa sta v vsakem koraku iz danega snarka izbrisala eno naključno povezavo in vozlišče ter
preverila, ali se je kromatični indeks snarka zmanjšal. Če se kromatični indeks snarka ni 
spremenil, je program iz novonastalega grafa izbrisal še eno naključno povezavo ali vozlišče. To je 
ponavljal, dokler kromatični indeks grafa ni bil enak 3. Ker je program odstranjeval naključne
povezave, je bilo treba to kodo večkrat ponoviti in izbrali najmanjše število povezav ali vozlišč izmed
izračunanih, da je vrednost odpornosti natančna. 
Nato sva primerjali odpornost vozlišč in odpornost povezav danih snarkov. Na koncu sva s poskušanjem 
poiskali najmanjši snark za katerega velja, da ima $vr(G) \ge 3$.





\end{document}