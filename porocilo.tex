\documentclass[a4paper, 10pt]{article}


\usepackage[slovene]{babel}
\usepackage[utf8]{inputenc}
\usepackage[T1]{fontenc}
\usepackage{amsmath}
\usepackage{graphicx}
\usepackage{listings}



\begin{document}
\newtheorem{example}{Zgled}
\newtheorem{theorem}{Izrek}
\newtheorem{proof}{Dokaz}
\newtheorem{definition}{Definicija}
\newtheorem{lemma}{Lema}

\thispagestyle{empty}
{\large
\noindent Univerza v Ljubljani\\[1mm]
Fakulteta za matematiko in fiziko\\[5mm]}
\vfill

\begin{center}{\Large
{\bf Odpornost snarkov}\\[2mm]
Vanja Kalaković in Eva Strašek\\[10mm]
Mentorja: Janoš Vidali, Riste Škrekovski\\[2mm]
Predmet: Finančna matematika \\[2mm]}
\end{center}
\vfill

{\large
Ljubljana, 2023}
\pagebreak



\section{Uvod}

Najprej bomo opredelili, kaj snarki sploh so.

\begin{definition}
    Če so vsa vozlišča grafa G enake stopnje k, pravimo, da je graf
k-regularen; 3-regularnim grafom pravimo tudi kubični grafi.
\end{definition}
\begin{definition}
    Kromatični indeks grafa $G$ je število $\chi'(G)$, ki predstavlja najmanjše 
    število barv, ki jih potrebujemo za barvanje povezav grafa G tako,
    da so sosednje povezave pobarvane različno.
\end{definition}
\begin{definition}
    Snark je ciklično 4-povezan kubičen graf z notranjim obsegom vsaj 5 in 
    kromatičnim indeksom 4. Snark reda $n$ in velikosti $m$ označimo z $S(n,m)$.
\end{definition}

 
\subsection*{Zgodovina snarkov}
Snark je svoje ime dobil leta 1976, ko ga je Markin Gardner poimenoval po skrivnostnem 
predmetu iz pesmi The Hunting of the Snark Lewisa Carrolla. \\
Prvi znani snark, Petersenov graf, je odkril Julius Petersen in zanj velja, da ima 10 vozlišč
in 15 povezav ter je najmanjši snark. V naslednjih letih so matematiki odrili nekaj manjših
snarkov, dokler ni do leta 1975 ameriški matematik Isaacs posplošil metodo, hrvaškega
matematika Blanuša, za izgradnjo dveh neskončnih družin snarkov, in sicer za družino
snarkov v obliki cvetja in za Blanuša-Descartes-Szekersov snark. Isaacs je odkril tudi snark 
poimenovan dvojna zvezda.  

\subsection*{Odpornost povezav in vozlišč}

\begin{theorem}{Vizingov izrek}
    Za enostaven graf G z maksimalno stopnjo $\Delta$(G) velja
    $\Delta$(G) $\leq \chi$'(G) $\leq \Delta$(G) + 1. 
\end{theorem}

\begin{definition}
    Snark je kritičen, če zanj velja, da če snarku odstranimo dve sosednji vozlišči postane 3-robno pobarljiv. 
\end{definition}

\begin{definition}
    Snark je bikritičen, če zanj velja, da če snarku odstranimo dve poljubni vozlišči postane 3-robno pobarljiv. 
\end{definition}

\begin{theorem}
    Če je snark bikritičen, zanj velja, da je tudi kritičen.
\end{theorem}

\begin{definition}
    Predpostavimo, da imamo snark $G = (V,E)$. Odpornost povezav $er(G)$ je je najmanjše 
    število $k$ povezav, ki jih je treba odstraniti, da postane snark $G$ 3-robno obarljiv. 
\end{definition}
\pagebreak
Koda, s katero sma računali odpornost povezav snarkov:
\begin{lstlisting}
    def odpornost_povezav(snark):
        snark = Graph(snark)
        seznam_povezav = snark.edges()
        st_povezav = snark.size()
        for i in range(st_povezav-3):
            combo = Combinations(seznam_povezav,i).list()
            for kombinacija in combo:
                graf = snark.copy()
                graf.delete_edges(kombinacija)
                if graf.chromatic_index() == 3:
                    return(i)
\end{lstlisting}


\begin{definition}
    Predpostavimo, da imamo snark $G = (V,E)$. Odpornost vozlišč $vr(G)$ je najmanjše število $k$
    vozlišč $E$, ki jih je treba odstraniti, da postane graf $G$ 3-robno obarljiv.
\end{definition}

Koda, s katero sma izračunali odpornost vozlišč snarkov: 
\begin{lstlisting}
    def odpornost_vozlisc(snark):
        st_vozlisc = Graph(snark).order()
        snark = Graph(snark)
        for i in range(st_vozlisc-3):
            combo = Combinations(range(st_vozlisc-1),i).list() 
            for kombinacija in combo:
                graf = snark.copy()
                graf.delete_vertices(kombinacija)
                if graf.chromatic_index() == 3:
                    return(i)
\end{lstlisting}

\subsection*{Predstavitev rezultatov}

V spodnji tabeli vidimo odpornosti povezav in odpornosti vozlišč kot sva
jih izračunali s pomočjo najine kode.

\begin{center}
    \begin{tabular}{|c|c|c|c|c|}
        \hline
        št. vozl. & št. povez. & odpor. povez. & odpor. vozl. \\
        \hline
        10 & 15 & 2 & 2 \\ 
        \hline
        18 & 27 & 2 & 2 \\
        \hline
        20 & 30 & 2 & 2 \\
        \hline
        22 & 33 & 2 & 2 \\
        \hline
        24 & 36 & 2 & 2 \\
        \hline
        26 & 39 & 2 & 2\\
        \hline 
        28 & 42 & 2 & 2 \\
        \hline
        30 & 45 & 2 & 2 \\
        \hline
        32 & 48 & 2 & 2 \\
        \hline
        34 & 51 & 2 & 2 \\
        \hline
        36 & 54 & 2 & 2 \\
        \hline
        38 & 57 & 2 & 2 \\
        \hline
        40 & 60 & 2 & 2 \\
        \hline
        44 & 66 & $ \ge 2$ & $ \ge 2$ \\
        \hline
    \end{tabular}
    \end{center}

Ugotovili sva, da velja da za vsak poljuben snark $G$ velja, da je njegova $vr(G) \ge 2$ in
njegova $er(G) \ge 2$. Poleg tega velja $er(G) = vr(G) $ Ugotovili sva, da so najmanjši
snarki, katerih $vr(G) \ge 3$ snarki z 44 vozlišči in 66 povezavami. Torej velja, da so vsi snarki,
ki imajo manj kot 44 vozlišč krtični ali bikritični.




\pagebreak
\subsection*{Potek dela:}
Najprej sva napisali program, ki je izračunal odpornost vozlišč $vr(G)$ in program, ki je 
izračunal odpornost povezav $er(G)$ za snarke, ki sva jih prenesli iz House of Graphs.
Programa sta v prvem koraku iz danega snarka izbrisala eno povezavo in vozlišče ter
preverila, ali se je kromatični indeks snarka zmanjšal. Če se kromatični indeks snarka ni 
spremenil, je program naredil seznam vseh možnih kombinacij dveh povezav in je za vsak element tega
seznama preveri ali se kromatični indeks spremeni, če ga odstranimo iz originalnega snarka. Če se 
kromatični indeks ne spremeni, program naredi seznam vseh možnih kombinacij treh povezav in treh vozlišč
ter za vsak element tega seznama preveri, ali kromatični indeks ostane enak, če odstranimo ta element.
To ponavlja, dokler kromatični indeks grafa ni enak 3. Ker je program preveril spremembo kromatičnega
indeksa za vse možne kombinacije vozlišč in povezav, je bila njegova časovna zahtevnost zelo velika
in se je v primeru, da smo ga uporabili na večjih snarkih ali večjem številu snarkov, izvajal vsaj nekaj ur.\\
Na koncu sva primerjali odpornost vozlišč in odpornost povezav danih snarkov. Na koncu sva s poskušanjem 
poiskali najmanjši snark za katerega velja, da ima $vr(G) \ge 3$.





\end{document}